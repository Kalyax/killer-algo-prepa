\documentclass[a4paper,10pt]{article}
\usepackage[french]{babel}
\usepackage[T1]{fontenc}
\usepackage{graphicx}

\newcounter{numcards}

\title{Algorithme du Killer}
\author{ }
\date{ }

\begin{document} 
\maketitle

Vous pouvez trouver la dernière version de l'algorithme ainsi 
que l'ensemble des fichiers à l'adresse :
\\{https://github.com/Kalyax/killer-algo-prepa}
\\(vous pouvez aussi l'utiliser pour me contacter si besoin)
\newline

Vous trouverez dans ce dossier l'algorithme du Killer, une explication pour l'utiliser (ce PDF) et un dossier "pdf" avec les règles.
Pensez à modifier les règles pour rajouter les jursiprudences de l'année dernière si besoin.
\newline

Pour utiliser l'algorithme vous avez besoin de Python 3 et d'une connection internet.

\section{Configuration de l'algorithme}


Rennomez le fichier "config\_exemple.txt" en "config.txt".
Le fichier se présente sous forme de tableau à deux colonnes "Classe" et "Joueur". 
Chaque retour à la ligne represente une ligne du tableau et on sépare la classe et le nom du joueur sur une même ligne par une virgule.
\newline

Vous trouverez à l'intérieur une liste de faux noms pour vous donnez un exemple. 
Commencez par tous les supprimer (sauf la première ligne avec Classe et Joueur) pour ensuite ajouter vos joueurs ainsi que leur classe.

\section{Lancer l'algorithme}

Assurez vous que le fichier "config.txt" soit dans le même dossier que l'algorithme (c'est à dire "killer.py").
\newline

Si vous êtes sur Windows, lancez le fichier "lancer\_windows.bat" en double cliquant dessus.

Si vous êtes sur Linux ou MacOS, ouvrez un terminal et placez vous dans le répertoire "killer-algo-prepa" (en utilisant la commande cd).
Exécutez en suite la commande "sh lancer\_mac\_linux.sh".
\newline

Un fichier "boucle.txt" devrait se créer ainsi qu'un fichier "cartes.tex".
Le premier contient la boucle où chaque joueur à une cible assignée. 
Conservez précieusement ce fichier.

\section{Générer les cartes}

Si vous avez LaTeX sur votre ordinateur, générez les cartes à l'aide du fichier "cartes.tex" généré par l'algorithme.
(Notez que le fichier "logo.png" doit être dans le même dossier lors de la génération)
\newline

Sinon, créez vous un compte sur https://www.overleaf.com/login.
Une fois connecté, créez un nouveau projet vide ("Blank project").
Importez les fichiers "cartes.tex" et "logo.png" dans l'éditeur avec le boutton "Upload" en haut
à gauche de l'écran (une sorte de rectangle avec une flèche vers le haut).
Vous pouvez soit cliquer sur "Select files" et sélectionner les fichiers ou bien vous pouvez les glisser sur la fenêtre.
\newline

Cliquez alors sur le fichier "cartes.tex" à gauche de votre écran.
Dans la deuxième fenêtre à droite, vous devriez avoir un boutton vert "Recompile".
Cliquez dessus et vos cartes se genèreront. Il ne vous reste plus qu'à les télécharger avec
le deuxième boutton juste à droite de "Recompile".


\section{Imprimer les cartes}

Une fois le PDF téléchargé, il faut l'imprimer en recto-verso en activant l'option 
"retourner sur les bords longs"
pour ne pas avoir de problèmes lors de l'impression.
Une fois imprimé, il est souhaitable de supprimer les fichiers "cartes.pdf" et "cartes.tex" pour des
questions de sécurité. Il est toujours possible de les regénérer avec le script "latex.py" si le fichier "boucle.txt"
se trouve dans le même dossier que ce dernier. 
(en utilisant les script "cartes\_windows.bat" ou "cartes\_mac\_linux.sh" prévus à cet effet)
\newline

Vous avez aussi un fichier "cartes\_arbitre.pdf" à disposition dans le dossier "pdf" avec des cartes d'arbitres que vous pouvez distribuer.
\newline

Il ne vous reste plus qu'à couper les cartes et les distribuer. Bon jeu!

\section{Chiffrage de la boucle}

Suite aux incidents de 2024, vous avez désormais un moyen de chiffrer la boucle si vous le souhaitez.
Cette partie est optionelle mais recommandée pour des questions de sécurité pour le bon déroulement du jeu.
La méthode possède des failles mais rajoute une couche de sécurité.
\newline

Cette sécurité nécéssite deux personnes : le maître du jeu qui possède l'algorithme et la boucle chiffrée sur
son ordinateur et l'arbitre principal qui est le seul à connaître le mot de passe. En aucun cas l'un doit posseder les deux.
Il est préférable que ces deux personnes ne se connaissent pas forcément au préalable pour éviter la triche.
\newline

Vous disposez de deux script "chiffrage\_windows.bat" et "chiffrage\_mac\_linux.sh" selon 
votre platforme pour chiffrer la boucle.
Ils s'éxécutent de la même manière que décrit précédament.
\newline

Le script commence par vous demander un mot de passe. Si vous chiffrez pour la première fois la boucle, inventez un mot de passe.
Si vous souhaitez déchiffrer, renseignez le mot de passe utilisé lors du chiffrement.
Vous avez alors le choix entre les deux options en écrivant "1" ou "2" comme expliqué lors de l'exécution du script.
\newline

Si vous chiffrez, le fichier "boucle.txt" doit être présent dans le dossier et un fichier "encrypt.txt" sera créé et le fichier "boucle.txt" SERA SUPPRIME. Il faut donc faire attention de
ne pas supprimer le fichier "encrypt.txt" pour ne pas perdre la boucle.
\newline

Si vous déchiffrez, le fichier "encrypt.txt" doit être présent dans le dossier et un fichier "boucle.txt" sera créé.
Il est souhaitable de supprimer le fichier "boucle.txt" après toute utilisation par les arbitres pour
que personne n'ait accès à la boucle sans la réunion du maître du jeu et de l'arbitre principal.



\section{Ancien algorithme}

En cas de besoin, vous trouverez l'ancien algorithme dans le fichier "ancien\_algorithme".


\end{document}